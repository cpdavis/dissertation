\subsubsection{Exploratory Order Analyses}
An interesting feature of this experimental design is that both manipulations (learning type and category type) push individuals towards a certain category learning system. Supervised and sparse blocks encourage use of the hypothesis-testing system, while unsupervised and dense blocks evoke the associative system. Thus, mismatch blocks (i.e., unsupervised-sparse and supervised-dense) have conflicting information on which category learning system to use and thus likely are less effective at evoking that system. To investigate this possibility, I completed some exploratory analyses.\par
	First, I compared unsupervised-dense blocks completed by groups 2 and 4. Group 2 completed a supervised-sparse (matching, hypothesis-testing) block before their supervised-dense block, while group 4 completed a supervised-dense (mismatch, hypothesis-testing) block before their unsupervised-dense block. If matching blocks more strongly evoke the category learning system and the hypothesized order effect (where activating the hypothesis-testing system first interferes with later use of the associative system) holds, then performance in group 2 on the unsupervised-dense block should be worse than performance in group 4 on the same block. A two-sample \textit{t}-test indicated that this hypothesis did not hold -- the two groups had equivalent performance (\textit{t}(73) = -0.62, \textit{p} = 0.54).\par
	I extended this analysis by doing the sample thing for groups 1 and 5, who both completed the supervised-sparse block second. Group 1 completed an unsupervised-dense (match, associative) block before their supervised-sparse block and group 4 completed an unsupervised-sparse (mismatch, associative) block first. Again, a two-sample \textit{t}-test indicated that the two groups had equivalent performance (\textit{t}(69) = 0.36, \textit{p} = 0.71). \par
	As an additional check, I looked at two more comparisons. First, I compared the unsupervised-dense blocks for the two groups who completed it first, group 1 and group 3. There should be no difference between these groups on this block, since it was the first block each group encountered. A two-sample \textit{t}-test confirmed this hypothesis (\textit{t}(70) = 0.097, \textit{p} = 0.92). I then checked the same thing for the supervised-sparse blocks within groups 2 and 6. Interestingly, these two groups were found to be different (\textit{t}(54) = -3.43, \textit{p} = 0.001).